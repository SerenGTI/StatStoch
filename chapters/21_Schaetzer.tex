%!TEX root = ../main.tex
\chapter{Schätzer}
\section{Allgemeines zum Schätzen}
Beim Schätzen wollen wir von einer Beobachtung einer (teilweise) unbekannten Verteilung auf Parameter der Verteilung schließen.

Das kann bedeuten, dass man versucht aus einer Stichprobe zu folgern, ob es sich zum Beispiel um eine Normalverteilung handelt.
Aber ebenso kann man zum Beispiel den Parameter $\mu$ der Normalverteilung schätzen.

Man sieht sofort dass man beim Schätzen einige Angaben entweder wissen oder Annahmen treffen muss.
Selbstverständlich kann man nur Parameter einer Verteilung schätzen wenn man annimmt dass es sich um diese Verteilung handelt.

Es gibt verschiede Arten zu Schätzen, so kann man beispielsweise einen Bereich angeben in dem sich der Parameter mit einer gewissen Sicherheit befindet (Lage- oder Intervallschätzer).




\section{Intervallschätzer}


Konfidenzintervalle
\subsection{Mit Tschebyscheff-Ungleichung}

\subsection{Über die Verteilungsfunktion}





\section{Maximum-Likelihood-Schätzer}
Die Maximum-Likelihood-Schätzmethode verwendet eine sogenannte Likelihoodfunktion $L$. Diese ist abhängig vom zu schätzenden Parameter $\alpha$, man sucht dann den Wert für $\alpha$ so, dass $L(\alpha)$ ein Maximum annimmt. Häufig wird dieser Wert mit einem Dach gekennzeichnet. 

Das Ergebnis dieses Schätzers wäre also allgemein
\begin{equation*}
	\hat\alpha = \underset\alpha\argmax L(\alpha)
\end{equation*}

Hierbei hängt die Likelihoodfunktion davon ab, was bei welcher Verteilung geschätzt wird. 


Wir wollen den Maximum-Likelihood-Schätzer anhand der Binomialverteilung betrachten.
Hierbei macht nur Schätzen des Parameters $p$ Sinn, denn selbst wenn nur eine einzige Stichprobe vorliegt, sieht man sofort welchen Wert $n$ hat.

Es liegt also eine Verteilung 
\begin{equation*}
	X\sim\binomial(n,p)
\end{equation*}
zugrunde, hierbei ist $n$ bekannt und fest, $p\in(0,1)$ wollen wir herausfinden.

Liegt uns nun eine Stichprobe $\omega$ vor, so ist auch $X(\omega)=k$ bekannt als die Anzahl Treffer in den $n$ Bernoulli-Experimenten.

Zum Schätzen fehlt uns noch die Likelihoodfunktion, diese ist hier
\begin{equation*}
	L:(0,1)\rightarrow[0,1], \enspace L(p)=P_p(X=k)=\binom{n}{k}*p^k(1-p)^{n-k}
\end{equation*}

Das heißt, es ist die Wahrscheinlichkeit dass $X$ mit Trefferwahrscheinlichkeit $p$ (als Parameter!) den Wert $k$ annimmt. Man sieht, dass diese Funktion ihr Maximum bei dem Wert für $p$ annimmt, unter dem es am wahrscheinlichsten ist, die Stichprobe $\omega$ zu erhalten.

Den Wert $\hat p$ zu berechnen würde hier durch Ableiten und anschließendem Bestimmen der Nullstelle erfolgen
\begin{equation*}
	\frac{\diff}{\diff p}L(\hat p)\overset!= 0.
\end{equation*}

\paragraph{Beispiel:}
Wir wollen dies nun nochmal an einem Beispiel veranschaulichen. Wir schätzen wiederum $p$ einer Binomialverteilung, schließen damit später aber auf einen anderen Wert.
\begin{eBox}{}{}
	Um die Populationsgröße $N$ einer Fischart in einem See zu schätzen, gehen wir folgendermaßen vor: Wir markieren von dieser Art $m=50$ Fische in einem See, fangen später $n=200$ Stück und stellen fest, dass $k=16$ davon markiert sind.

	Wir nehmen an, dass das Untersuchen der Stichprobe auf Markierungen eine Folge von $n$ unabhängigen Bernoulli-Experimenten mit Trefferwahrscheinlichkeit $p=\frac mN$ ist. Also beschreibt $p$ den Anteil der markierten Fische im See.
\end{eBox}
Wir verwenden die zuvor besprochene Likelihoodfunktion $L(p)$, denn mit $p=\frac mN$ können wir auf $N$ schließen. 

Sei $X$ eine Zufallsvariable die die Anzahl markierter Fische in der Stichprobe beschreibt, damit ist $X \sim \binomial(200,p)$ verteilt. 

Die Likelihoodfunktion ist mit diesen Werten
\begin{equation*}
	L(p)=P_p(X=k)=\binom{200}{k}* p^{k}(1-p)^{200-k}
\end{equation*}
Damit folgt
\begin{align*}
	\frac{\diff}{\diff p}L(p) &= \binom{200}{k}*\left[ k*p^{k-1}(1-p)^{200-k} - p^k*(200-k)*(1-p)^{200-k-1}\right]\\
	&=\underbrace{\binom{200}{16}* p^{k-1}(1-p)^{200-k-1}}_{>0} * \left[ k(1-p)-200p+kp \right]\\
\end{align*}
Da wir uns für Nullstellen interessieren, betrachten wir nur noch den rechten Teil und erhalten
\begin{align*}
	k(1-\hat p)-200\hat p+k\hat p&\overset != 0\\
	k-k\hat p-200\hat p+k\hat p &= 0\\
	k-200\hat p&=0
\end{align*}
Für unsere Stichprobe $\omega$ ist $X(\omega)=k=16$, damit ist
\begin{align*}
	\hat p =\frac{m}{N} =\frac k{200}\Rightarrow N&=\frac {m*200}{k}\\
	\Rightarrow N &= \frac{50*200}{16} = 625
\end{align*}

Wir haben damit die Populationsgröße der Fischart auf 625 geschätzt.

\section{Kleinste Quadrate-Schätzer}


\section{Bayes-Schätzer}