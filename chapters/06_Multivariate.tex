\chapter{Multivariate}
Bisher wurden nur eindimensionale Daten erfasst, nicht aber verschschiedene Merkmale in Zusammenhang gebracht und gemeinsam betrachtet oder miteinander verglichen.
\section{Kontingenztafel}
Die Kontingenztabelle eignet sich zu Darstellung der gemeinsamen Verteilung von zwei Diskreten Merkmalen mit relativ wenigen Ausprägungen.

Auf Basis der Ausprägungen $a_1,\ldots,a_k$ des Merkmals $X$ und $b_1,\ldots,b_m$ für $Y$ liegen in der Urliste die gemeinsamen Messwerte vor. Das heißt die Urliste besteht aus den Tupeln $(a_i,b_j)$. Analog zum Eindimensionalen sind die absoluten Häufigkeiten $h_{ij}$ definiert. Darauf aufbauend ebenfalls völlig analog die relativen Häufigkeiten $f_{ij}$.

\paragraph{Kontingenztafel der absoluten Häufigkeiten}
Die aus diesen Werten entstehende Tafel heißt $(k\times m)$-Kontingenztafel der absoluten Häufigkeiten. Sie enthält neben den Häufigkeitsdaten zusätzlich noch die Spalten- beziehungsweise Zeilensummen der Werte.

\begin{center}
	\begin{tabular}{c|ccc|c}
		& $b_1$ & $\cdots$ & $b_m$ &\\
		\hline $a_1$ & $h_{11}$ & $\cdots$ & $h_{1m}$&$h_{1\cdot}=\sum_{i=1}^m h_{1i}$\\
		$a_2$ & $h_{21}$ & $\cdots$ & $h_{2m}$&$h_{2\cdot}=\sum_{i=1}^m h_{2i}$\\
		$\vdots$ & $\vdots$ & $\cdots$ & $\vdots$&\\
		$a_1$ & $h_{k1}$ & $\cdots$ & $h_{km}$&$h_{k\cdot}=\sum_{i=1}^m h_{ki}$\\
		\hline &$h_{\cdot1}$&&$h_{\cdot m}$&$n$
	\end{tabular}
\end{center}

Die Zeilensummen $h_{i\cdot}$ werden auch als Randhäufigkeiten des Merkmals $X$ bezeichnet. Diese Werte sind die einfachen Häufigkeiten mit denen das Merkmal $X$ die Werte $a_1,\ldots,a_k$ annimmt, wenn $Y$ nicht berücksichtigt wird.

Analog dazu sind die Spaltensummen die Häufigkeiten von $Y$ unter Vernachlässigung des Merkmals $X$.

\paragraph{Kontingenztafel der relativen Häufigkeiten}
Da Anteile beziehungsweise Prozente häufig anschaulicher sind als absolute Häufigkeitswerte betrachtet man häufig auch die Häufigkeitstafel der relativen Häufigkeiten. Diese entsteht durch teilen durch die Gesamtzahl $n$.

\begin{center}
	\begin{tabular}{c|ccc|c}
		& $b_1$ & $\cdots$ & $b_m$ &\\
		\hline $a_1$ & $h_{11}$ & $\cdots$ & $h_{1m}$&$f_{1\cdot}=\sum_{i=1}^m f_{1i}$\\
		$a_2$ & $f_{21}$ & $\cdots$ & $f_{2m}$&$f_{2\cdot}=\sum_{i=1}^m f_{2i}$\\
		$\vdots$ & $\vdots$ & $\cdots$ & $\vdots$&\\
		$a_1$ & $f_{k1}$ & $\cdots$ & $f_{km}$&$f_{k\cdot}=\sum_{i=1}^m f_{ki}$\\
		\hline &$f_{\cdot1}$&&$f_{\cdot m}$&$1$
	\end{tabular}
\end{center}

\section{Bedingte Häufigkeiten}
S125/110