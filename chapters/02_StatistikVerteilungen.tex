\chapter{Verteilungen und ihre Darstellungen}
\section{Häufigkeiten}
Als \emph{Urliste} bezeichnet man die Menge der Merkmale $X$ der Untersuchungseinheiten $U=\simpleset{x_1,\ldots,x_n}$.

Die \emph{auftretenden Ausprägungen} von $X$ sind die Werte $\simpleset{a_1,\ldots,x_n}\subseteq\simpleset{x_1,\ldots,x_n}, k\leq n$.
Oftmals treten in einem großen Datensatz der Größe $n$ nicht auch $n$ verschiedene Werte $x_i$ auf.

Damit definieren sich 
\begin{definition}{Absolute Häufigkeit}
	Die absolute Häufigkeit einer auftretenden Ausprägung $a$ in einer Urliste $U$ ist
	\begin{equation*}
		h(a)=\left|\set{i\in\N}{x_i=a, x_i\in U}\right|.
	\end{equation*}
\end{definition}
Es gilt immer, dass die Summe aller absoluten Häufigkeiten gleich der Datensatzgröße ist
\begin{equation*}
	\sum\limits_{i=1}^n h(a_i)=|U|.
\end{equation*}

Die absolute Häufigkeitsverteilung ist dargestellt durch die Folge von Werten
$$h_1,\ldots,h_k=h(a_i),\ldots,h(a_k)$$

\begin{definition}{Relative Häufigkeit}
	Die relative Häufigkeit einer auftretenden Ausprägung $a$ in einer Urliste $U$ ist
	\begin{equation*}
		f(a)=\frac{h(a)}{|U|}.
	\end{equation*}
\end{definition}
Es gilt ähnlich wie bei der absoluten Häufigkeit für die Summe
\begin{equation*}
	\sum\limits_{i=1}^n f(a_i)=1.
\end{equation*}


Eine grafische Darstellung einer Häufigkeitsverteilung nennt man ein \emph{Histogramm}. Bei Histogrammen ist auf die Flächentreue zu achten, das bedeutet, dass der Flächeninhalt der aufgetragenen Rechtecke proportional (oder gleich) zu $h_j$ oder $f_j$ ist. So kann das menschliche Auge die Verteilung besser wahrnehmen.

Hat das Histogramm einer Verteilung nur einen deutlich erkennbaren Hochpunkt (Gipfel), heißt sie \emph{unimodal}. Treten mehrere Gipfel auf nennt man die Verteilung \emph{multimodal}. Bei zwei Gipfeln spricht man von einer \emph{bimodalen} Verteilung.

Man nennt eine Verteilung \emph{symmetrisch}, wenn es eine Symmetrieachse gibt, sodass die rechte und linke Hälfte der Verteilung annähernd zueinander spiegelbildlich sind. 
Eine Verteilung heißt \emph{schief}, wenn sie deutlich unsymmetrisch ist. Sie heißt dann \emph{linkssteil oder rechtsschief}, wenn der überwiegende Anteil von Daten linksseitig konzentriert ist. Dann steigt die Verteilung links deutlich steiler ab als rechts. Entsprechend \emph{rechtssteile oder linksschiefe} Verteilungen.


\section{Kumulierte Häufigkeiten}
Die kumulierten Häufigkeitsverteilungen geben an, wie viele Datenpunkte der Urliste, beziehungsweise welcher Anteil der Daten unterhalb einer Schranke liegen. Um diese Aussage sinnvoll zu beantworten ist zumindest eine Ordinalskala nötig.

\begin{definition}{Absolute kumulierte Häufigkeitsverteilung}
	Die absolut kumulierte Häufigkeitsverteilung ist die Funktion
	\begin{equation*}
		H(x)=\sum\limits_{i:a_i\leq x} h_i.
	\end{equation*}
\end{definition}
\begin{definition}{Relative kumulierte Häufigkeitsverteilung}
	Die relative kumulierte Häufigkeitsverteilung oder auch empirische Verteilungsfunktion ist
	\begin{equation*}
		F(x)=\sum\limits_{i:a_i\leq x} f_i.
	\end{equation*}
\end{definition}
Die kumulierten Häufigkeitsverteilungen sind monoton wachsende Treppenfunktionen, die an den Sprungstellen rechtsseitig stetig sind.



\section{Gruppierung}
Sind alle auftretenden Ausprägungen Elemente eines Interfalls $[a,b]$, lässt sich dieses in gleich große Klassen der Größe $d$ unterteilen.

Eine Klassifizierung ist allgemein
\begin{equation*}
	[a, c_1), \ldots, [c_i,c_{i+1}), [c_{i+1},c_{i+2}), \ldots\quad \forall i: c_{i+1}-c_i=d
\end{equation*}

Klassifizierte Daten sind i.A. einfacher zu interpretieren als große Mengen von Daten, die sich nur wenig voneinander unterscheiden.

Der maximale Fehler bei der Klassifizierung ist die halbe Klassengröße.




