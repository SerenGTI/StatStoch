\chapter{Themenbereiche}
\section{Desktriptive Statistik}

\section{Explorative Statistik}

\section{Induktive Statistik}





\chapter{Grundbegriffe}
\section{Grundbegriffe der Statistik}
\begin{description}
	\item[Statistische Einheit]
	Objekte die erfasst werden

	\item[Grundgesamtheit]
	Menge aller für die Fragestellung relevanten statistischen Einheiten

	\item[Teilgesamtheit]
	Teilmenge der Grundgesamtheit

	\item[Stichprobe]
	Tatsächlich untersuchte Teilmenge der Grundgesamtheit

	\item[Merkmal, Variable]
	Größe von Interesse

	\item[Ausprägung, Wert]
	Konkreter Wert des Merkmals für eine statistische Einheit
\end{description}
\section{Charakterisierung der Merkmale}
\begin{description}
	\item[diskret]
	abzählbar
	\item[stetig]
	Werte aus einem Intervall
	\item[quasi-stetig]
	stetig, aber nicht stetig messbar
\end{description}
\section{Skalen von Merkmalen}
\begin{description}
	\item[Nominalskala]
		Namen, Kategorien
	\item[Ordinalskala]
		Ausprägungen mit Ordnung, aber Abstände nicht interpretierbar
	\item[Kardinalskala]
		metrisch, messbar
\end{description}

\section{Datengewinnung, Datenerhebung}
Experiment - Erhebung


