\chapter{Statistik}
\section{Erste Kenngrößen}
Als \emph{Urliste} bezeichnet man die Menge der Merkmale $X$ der Untersuchungseinheiten $U=\simpleset{x_1,\ldots,x_n}$.

Die \emph{auftretenden Ausprägungen} von $X$ sind die Werte $\simpleset{a_1,\ldots,x_n}\subseteq\simpleset{x_1,\ldots,x_n}, k\leq n$.
Oftmals treten in einem großen Datensatz der Größe $n$ nicht auch $n$ verschiedene Werte $x_i$ auf.

Damit definieren sich 
\begin{definition}{Absolute Häufigkeit}
	Die absolute Häufigkeit einer auftretenden Ausprägung $a$ in einer Urliste $U$ ist
	\begin{equation*}
		h(a)=\left|\set{x\in U}{x=a}\right|.
	\end{equation*}
\end{definition}
Es gilt immer, dass die Summe aller absoluten Häufigkeiten gleich der Datensatzgröße ist
\begin{equation*}
	\sum\limits_{i=1}^n h(a_i)=|U|.
\end{equation*}

Die absolute Häufigkeitsverteilung ist dargestellt durch die Folge von Werten
$$h_1,\ldots,h_k=h(a_i),\ldots,h(a_k)$$


Eine grafische Darstellung der absoluten Häufigkeitsverteilung nennt man ein \emph{Histogramm}.

\begin{definition}{Relative Häufigkeit}
	Die relative Häufigkeit einer auftretenden Ausprägung $a$ in einer Urliste $U$ ist
	\begin{equation*}
		f(a)=\frac{h(a)}{|U|}.
	\end{equation*}
\end{definition}
Es gilt ähnlich wie bei der absoluten Häufigkeit für die Summe
\begin{equation*}
	\sum\limits_{i=1}^n f(a_i)=1.
\end{equation*}

\subsection{Kumulierte Häufigkeiten}
Die absolut kumulierte Häufigkeitsverteilung ist die Funktion
\begin{equation*}
	H(x)=\sum\limits_{i:a_i\leq x} h_i
\end{equation*}
Ebenso, die relative kumulierte Häufigkeitsverteilung
\begin{equation*}
	F(x)=\sum\limits_{i:a_i\leq x} f_i
\end{equation*}

\section{Klassifizierung}
Sind alle auftretenden Ausprägungen Elemente eines Interfalls $[a,b]$, lässt sich dieses in gleich große Klassen der Größe $d$ unterteilen.

Eine Klassifizierung ist allgemein
\begin{equation*}
	[a, c_1), \ldots, [c_i,c_{i+1}), [c_{i+1},c_{i+2}), \ldots\quad \forall i: c_{i+1}-c_i=d
\end{equation*}

Klassifizierte Daten sind i.A. einfacher zu interpretieren als große Mengen von Daten, die sich nur wenig voneinander unterscheiden.

Der maximale Fehler bei der Klassifizierung ist die halbe Klassengröße.

\section{Lagemaße}
Lagemaße helfen beim Vergleich verschiedener Eigenschaften, bzw dem Vergleich verschiedener statistischer Einheiten mit einer gemeinsamen Eigenschaft. 

\begin{definition}{Lagemaß}
	Ein \emph{Lagemaß} ist eine Abbildung $L:\R^n\rightarrow \R$ mit der Eigenschaft
	\begin{equation*}
		L(x_1+a,\ldots,x_n+a)=L(x_1,\ldots,x_n)+a\quad \forall a,x_i\in\R\enspace (1\leq i\leq n)
	\end{equation*}
	Ein Lagemaß beschreibt das Zentrum einer Verteilung.
\end{definition}

Beispiele für Lagemaße sind

\subsection{Arithmetisches Mittel}
Das arithmetische Mittel ist nur für quantitative Merkmale sinnvoll. Es berechnet sich durch
\begin{equation*}
    \overline x=\frac 1n \sum\limits_{i=1}^n x_i=\sum\limits_{i=1}^n (a_i*f_i)
\end{equation*}
aus Rohdaten beziehungsweise aus den Häufigkeitsdaten.

Mit dem arithmetischen Mittel gilt die sogenannte Schwerpunkteigenschaft
\begin{equation*}
	\sum\limits_{i=1}^n(x_i-\overline x)=0
\end{equation*}

Wie aus der Formel erkennbar, ist das arithmetische Mittel extrem empfindlich gegen Ausreißer. Dafür wurden die folgenden Mittel eingeführt.

\paragraph{Das getrimmte Mittel}
Um Ausreißer weniger stark ins Gewicht fallen zu lassen wird der Datensatz absichtlich verkleinert. Beim getrimmten Mittel aus einer sortiert vorliegenden Liste von Daten werden zum Beispiel die oberen und unteren $5\%$ der Daten abgeschnitten, damit fallen auch eventuelle Ausreißer raus. Die Datensatzgröße bleibt jedoch nicht erhalten.

\paragraph{Das winsorisierte Mittel}
Ähnlich wie beim getrimmten Mittel wird der Datensatz beim winsorisierten Mittel von oben und unten herein bearbeitet. Anstatt Daten zu löschen werden beispielsweise die oberen $5\%$ durch den nächstkleineren Wert ersetzt. Hierbei bleibt also die Datensatzgröße gleich.

\subsection{Median}
Der Median stellt ein robusteres Lagemaß als das arithmetische Mittel dar, er ist resistenter gegen Ausreißer im Datensatz.
Für $x_1\leq x_2\leq\ldots\leq x_n$, also einen sortiert vorliegenden Datensatz ist der Median
\begin{equation*}
	x_{\operatorname{med}}=
	\begin{cases}
		x_{\frac{n+1}2}& \text{$n$ ungerade}\\
		\frac 12 (x_{\frac n2}+x_{\frac n2 +1})& \text{$n$ gerade}\\
	\end{cases}
\end{equation*}
Benötigt eine Zahlenordnung (oridnal).

\subsection{Modus}

