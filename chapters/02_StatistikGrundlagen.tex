\chapter{Statistik}
\section{Erste Kenngrößen}
Als \emph{Urliste} bezeichnet man die Menge der Merkmale $X$ der Untersuchungseinheiten $U=\simpleset{x_1,\ldots,x_n}$.

Die \emph{auftretenden Ausprägungen} von $X$ sind die Werte $\simpleset{a_1,\ldots,x_n}\subseteq\simpleset{x_1,\ldots,x_n}, k\leq n$.
Oftmals treten in einem großen Datensatz der Größe $n$ nicht auch $n$ verschiedene Werte $x_i$ auf.

Damit definieren sich 
\begin{definition}{Absolute Häufigkeit}
	Die absolute Häufigkeit einer auftretenden Ausprägung $a$ in einer Urliste $U$ ist
	\begin{equation*}
		h(a)=\left|\set{x\in U}{x=a}\right|.
	\end{equation*}
\end{definition}
Es gilt immer, dass die Summe aller absoluten Häufigkeiten gleich der Datensatzgröße ist
\begin{equation*}
	\sum\limits_{i=1}^n h(a_i)=|U|.
\end{equation*}

Die absolute Häufigkeitsverteilung ist dargestellt durch die Folge von Werten
$$h_1,\ldots,h_k=h(a_i),\ldots,h(a_k)$$


Eine grafische Darstellung der absoluten Häufigkeitsverteilung nennt man ein \emph{Histogramm}.

\begin{definition}{Relative Häufigkeit}
	Die relative Häufigkeit einer auftretenden Ausprägung $a$ in einer Urliste $U$ ist
	\begin{equation*}
		f(a)=\frac{h(a)}{|U|}.
	\end{equation*}
\end{definition}
Es gilt ähnlich wie bei der absoluten Häufigkeit für die Summe
\begin{equation*}
	\sum\limits_{i=1}^n f(a_i)=1.
\end{equation*}

\subsection{Kumulierte Häufigkeiten}
Die absolut kumulierte Häufigkeitsverteilung ist die Funktion
\begin{equation*}
	H(x)=\sum\limits_{i:a_i\leq x} h_i
\end{equation*}
Ebenso, die relative kumulierte Häufigkeitsverteilung
\begin{equation*}
	F(x)=\sum\limits_{i:a_i\leq x} f_i
\end{equation*}

\section{Klassifizierung}
Sind alle auftretenden Ausprägungen Elemente eines Interfalls $[a,b]$, lässt sich dieses in gleich große Klassen der Größe $d$ unterteilen.

Eine Klassifizierung ist allgemein
\begin{equation*}
	[a, c_1), \ldots, [c_i,c_{i+1}), [c_{i+1},c_{i+2}), \ldots\quad \forall i: c_{i+1}-c_i=d
\end{equation*}

Klassifizierte Daten sind i.A. einfacher zu interpretieren als große Mengen von Daten, die sich nur wenig voneinander unterscheiden.

\section{Lagemaße}
