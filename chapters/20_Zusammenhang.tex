%!TEX root = ../main.tex
\chapter{Zusammenhang von Zufallsvariablen}
\section{Kovarianz}
Für diskrete Zufallsvariablen wurde die Unabhängigkeit von Zufallsvariablen definiert als
\begin{equation*}
	P(X=x,Y=y)=P(X=x)*P(Y=y)\quad\forall x,y\in\Omega.
\end{equation*}

Allerdings macht es auch Sinn für abhängige Zufallsvariablen zu untersuchen, wie sich die Abhängigkeit ausprägt.

\begin{definition}{Kovarianz}
	Die Kovarianz von zwei Zufallsvariablen $X$ und $Y$ ist
	\begin{align*}
		\cov(X,Y)&\coloneqq E[(X-E(X))*(Y-E(Y))]\\
		\text{(diskret)}\enspace&=\sum_i\sum_j (x_i-E(X))(y_j-E(Y))*f(x_i,y_j)\\
		\text{(stetig)}\enspace&=\int_{-\infty}^\infty\int_{-\infty}^\infty (x-E(X))(y-E(Y))*f(x,y)\intd y\intd x
	\end{align*}
\end{definition}
\begin{itemize}
	\item $\cov(X,Y)$ ist groß, wenn $X$ und $Y$ gleichförmig von den jeweiligen Erwartungswerten abweichen.
	\item Die Kovarianz ist negativ bei gegenläufiger Abweichung.
	\item Ist $\cov(X,Y)\approx 0$, so besteht keine besondere Wirkungsbeziehung zwischen den Variablen.
\end{itemize}

\begin{satz}{Eigenschaften der Kovarianz}%
	Für zwei Zufallsvariablen $X$ und $Y$ gilt immer
	\begin{enumerate}%
		\item Die Kovarianz ist symmetrisch, 
		\begin{equation*}
			\cov(X,Y)=\cov(Y,X).
		\end{equation*}
		\item Es gilt der Verschiebungssatz
		\begin{equation*}
			\cov(X,Y)=E(X*Y)-E(X)*E(Y)
		\end{equation*}
		Hieraus wird klar, dass wenn Variablen unabhängig sind, dann ihre Kovarianz immer $0$ ist. Ist eine Kovarianz $0$ hat das zunächst keine Aussage über die Unabhängigkeit.
		\item Die Kovarianz ist extrem maßstabsabhängig. Für linear transformierte $X$ und $Y$ ist
		\begin{equation*}
			\cov(aX+b,cY+d)=a*c*\cov(X,Y).
		\end{equation*}
		\item Für die Varianz der Addition zweier Zufallsvariablen gilt außerdem
		\begin{equation*}
			V(X+Y)=V(X)+V(Y)+2\cov(X,Y).
		\end{equation*}
	\end{enumerate}
\end{satz}
\paragraph{Beweis:}
\begin{enumerate}
	\item Ergibt sich direkt aus dem Erwartungswert.
	\item Lässt sich leicht durch Ausmultiplizieren sehen
	\begin{align*}%
		\cov(X,Y)&=E[(E-E(X))*(Y-E(Y))]\\
		&=E[X*Y-E(Y)*X-Y*E(X)+E(X)*E(Y)]\\
		&=E(X*Y)-E(X)*E(Y)
	\end{align*}
\end{enumerate}

\section{Korrelationskoeffizient}
Der Korrelationskoeffizient stellt ein maßstabsunabhängiges Zusammenhangsmaß dar.
\begin{equation*}
	\rho(X,Y)=\frac{\cov(X,Y)}{\sqrt{V(X)*V(Y)}}=\frac{\cov(X,Y)}{\sigma_X*\sigma_Y}
\end{equation*}
\begin{itemize}
	\item $-1\leq\rho\leq1$
\end{itemize}