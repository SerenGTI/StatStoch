%!TEX root = ../main.tex
\subsection{Hypergeometrische Verteilung}
Man betrachtet eine Grundgesamtheit der Größe $N$, in der $M$ Elemente eine gewünschte Eigenschaft besitzen, d.h. einen Treffer darstellen. 

Eine Zufallsvariable die bei einer Stichprobe der Größe $n$ (ohne Zurücklegen) die Anzahl Treffer zählt heißt dann \emph{hypergeometrisch verteilt}, man schreibt $X\sim \hypergeom(n,M,N-M)$.
\begin{equation*}
 	P(X=k)=\frac{\binom Mk*\binom{N-M}{n-k}}{\binom Nn}
\end{equation*}
Sei $A_j$ das Ereignis, dass die $j$te Kugel die gewünschte Eigenschaft hat.
Die Zufallsvariable lässt sich also schreiben als
\begin{equation*}
	X=\sum_{j=1}^n 1_{A_j}
\end{equation*}
Die Wahrscheinlichkeit, für $A_j$ ist $P(A_j)=\frac MN$. Damit ist der Erwartungswert einer hypergeometrisch verteilten Zufallsvariable
\begin{equation*}
	E(X)=E\Big(\sum_{j=1}^n 1_{A_j} \Big)=\sum_{j=1}^n E(1_{A_j})=n*\frac MN
\end{equation*}
