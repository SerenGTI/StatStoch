% \chapter{Themenbereiche}
% \section{Desktriptive Statistik}

% \section{Explorative Statistik}

% \section{Induktive Statistik}





\chapter{Grundbegriffe}
\section{Grundbegriffe der Statistik}
\begin{description}
	\item[Statistische Einheit]
	Objekte die erfasst werden und an denen die interessierenden Größen erfasst werden

	\item[Grundgesamtheit]
	Menge aller für die Fragestellung relevanten statistischen Einheiten

	\item[Teilgesamtheit]
	Teilmenge der Grundgesamtheit

	\item[Stichprobe]
	Tatsächlich untersuchte Teilmenge der Grundgesamtheit

	\item[Merkmal, Variable]
	Größe von Interesse

	\item[Merkmalsausprägung, Wert]
	Konkreter Wert des Merkmals für eine bestimmte statistische Einheit
\end{description}

\section{Charakterisierung der Merkmale}
\begin{description}
	\item[diskret]
	Merkmale, die nur endlich viele oder abzählbar unendlich viele Ausprägungen annehmen sind diskret.
	\item[stetig]
	Merkmale, die Werte aus einem Intervall annehmen können heißen stetig.
	\item[quasi-stetig]
	Merkmale, die sich nur diskret messen lassen aber aufgrund einer sehr feinen Abstufung wie stetige Merkamle behandelt werden können.
\end{description}

Die Ausprägungen eines stetigen Merkmals lassen sich immer so zusammenfassen, dass es als diskret angesehen werden kann. Die Ausprägungen heißen dann gruppiert oder klassiert.



\section{Skalen}
Zusätzlich zur Charakterisierung der Merkmale werden diese anhand ihres Skalenniveaus unterschieden.
\begin{description}
	\item[Nominalskala]
		Wenn die Ausprägungen Namen oder Kategorien sind, die den Einheiten zugeordnet werden heißt das Merkmal \emph{nominalskaliert}. Beispielsweise Geschlecht oder Verwendungszweck.
	\item[Ordinalskala]
		Merkmale mit Ausprägungen zwar mit Ordnung, bei denen allerdings ein Abstand der Merkamale nicht interpretier- oder vergleichbar ist heißen \emph{ordinalskaiert}. Ein Beispiel hierfür wären Schulnoten.
	\item[Kardinalskala]
		Ein kardinalskaliertes Merkmal wird oft auch metrisch bezeichnet. Hierbei sind die Abstände der Ausprägungen interpretierbar und zusätzlich ist ein sinnvoller Nullpunkt der Skala festgelegt oder bestimmbar.
\end{description}

Auf Basis dieser Skalenmerkmale nennt man Merkmale mit endlich vielen Ausprägungen, die höchstens ordinalskaliert sind \emph{qualitative} oder \emph{kategoriale Merkmale}. Diese geben eine Qualität aber nicht ein Ausmaß wieder. 

Geben die Ausprägungen jedoch eine Intensität oder Ausmaß wieder so spricht man von \emph{quantitativen Merkmalen}. Alle Messungen mit Zahlenwerten stellen Ausprägungen quantiativer Merkmale dar. Ein kardinalskaliertes Merkmal ist stets quantitativ.

% \section{Datengewinnung, Datenerhebung}
% S.18 ff


