%!TEX root = ../main.tex

\subsection{Exponentialverteilung}
Die Exponentialverteilung stellt das kontinuierliche Äquivalent zur geometrischen Verteilung dar.

Eine Zufallsvariable $X$ heißt \emph{exponentialverteilt} mit Parameter $\lambda$,  wenn für Dichte und Verteilung gilt
\begin{center}
	\begin{tikzpicture}[scale=0.7]
		\draw[->, line width=0.3mm] (0,4) to (6.5,4) node[below] {$x$};
		\draw[->, line width=0.3mm] (0,4) to (0,7) node[right] {$f^X(x)$};
		\draw[->, line width=0.3mm] (0,0) to (6.5,0) node[below] {$x$};
		\draw[->, line width=0.3mm] (0,0) to (0,3) node[right] {$F^X(x)$};

		\draw (0,6) node (eins) [rectangle,inner sep = 0pt,minimum size = 0pt,minimum width=4pt,draw, label={left:$\lambda$}] {};
		
		\draw[line width=0.5mm,blue] (-1,4) -- (0,4);
		\draw[line width=0.5mm,scale=1,domain=0:6,smooth,variable=\x,blue] plot ({\x},{2*exp(-(2*\x)/3)+4});
		\draw (0,6) node [fill = blue,circle,inner sep = 0pt,minimum size = 4pt] {};

		\draw[line width=0.5mm,blue] (-1,0) -- (0,0);
		\draw[line width=0.5mm,scale=1,domain=0:6,smooth,variable=\x,blue] plot ({\x},{2*(1-exp(-(4*\x)/3))});
		
		\draw (0,2) node (eins) [rectangle,inner sep = 0pt,minimum size = 0pt,minimum width=4pt,draw, label={left:$1$}] {};

		\draw (-6, 5.5) node {$f^X(x)=
			\begin{cases}
				0&x<0\\
				\lambda*e^{-\lambda*x}&\text{sonst}
			\end{cases}$};
		\draw (-6, 1.5) node {$F^X(x)=
			\begin{cases}
				0&x<0\\
				1-e^{-\lambda*x}&\text{sonst}
			\end{cases}$};
	\end{tikzpicture}%	
\end{center}%
Man schreibt dann $X\sim\expo(\lambda)$.

\paragraph{Erwartungswert und Varianz}
Für eine Zufallsvariable $Y\sim\expo(1)$ ist $E(Y)=V(Y)=1$. Damit lässt sich der allgemeine Fall einfach berechnen
\begin{align*}
	X\sim\expo(\lambda)\quad\rightarrow\quad E(X)&=E\left(\frac Y\lambda\right)= \frac1\lambda\\
	V(X)&=V\left(\frac Y\lambda\right)=\frac{1}{\lambda^2}.
\end{align*}
\begin{satz}{Gedächtnislosigkeit der Exponentialverteilung}
	Für eine exponentialverteilte Zufallsvariable $X\sim \expo(\lambda)$ und $x,y\in\R$ gilt
	\begin{equation*}
		P(X\geq x+y\,|\,X\geq x)=P(X\geq y)
	\end{equation*}
	das heißt, $X$ ist gedächtnislos. Dies ist analog zur geometrischen Verteilung.
\end{satz}
\paragraph{Beweis:}
\begin{align*}
	P(X\geq x+y\,|\,X\geq x)&=\frac{P(X\geq x+y\wedge X\geq x)}{P(X\geq x)}=\frac{P(X\geq x+y)}{P(X\geq x)}\\
	&=\frac{1-F^X(x+y)}{1-F^X(x)}\\
	&=\exp(-\lambda(x+y)-(-\lambda x))=\exp(-\lambda y)\\
	&=1-F^X(y)=P(X\geq y)
\end{align*}
