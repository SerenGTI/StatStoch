\section{Lagemaße}
Lagemaße helfen beim Vergleich verschiedener Eigenschaften, bzw dem Vergleich verschiedener statistischer Einheiten mit einer gemeinsamen Eigenschaft. 

\begin{definition}{Lagemaß}
	Ein \emph{Lagemaß} ist eine Abbildung $L:\R^n\rightarrow \R$ mit der Eigenschaft
	\begin{equation*}
		L(x_1+a,\ldots,x_n+a)=L(x_1,\ldots,x_n)+a\quad \forall a,x_i\in\R\enspace (1\leq i\leq n)
	\end{equation*}
	Ein Lagemaß beschreibt das Zentrum einer Verteilung.
\end{definition}

Beispiele für Lagemaße sind

\subsection{Arithmetisches Mittel}
Das arithmetische Mittel ist nur für quantitative Merkmale sinnvoll. Es berechnet sich durch
\begin{equation*}
    \overline x=\frac 1n \sum\limits_{i=1}^n x_i=\sum\limits_{i=1}^n (a_i*f_i)
\end{equation*}
aus Rohdaten beziehungsweise aus den Häufigkeitsdaten.

Mit dem arithmetischen Mittel gilt die sogenannte \emph{Schwerpunkteigenschaft}
\begin{equation*}
	\sum\limits_{i=1}^n(x_i-\overline x)=0.
\end{equation*}

Unter einer linearen Transformation $x\mapsto ax+b$ verhält sich das arithmetische Mittel analog
\begin{equation*}
    \overline x =\frac 1n\sum_{i=1}^n (ax_i+b)=\frac an\sum_{i=1}^n x_i+b=a\overline x +b
\end{equation*}


Wie aus der Formel erkennbar, ist das arithmetische Mittel extrem empfindlich gegen Ausreißer. Dafür wurden die folgenden Mittel eingeführt.

\paragraph{Das getrimmte Mittel}
Um Ausreißer weniger stark ins Gewicht fallen zu lassen wird der Datensatz absichtlich verkleinert. Beim getrimmten Mittel aus einer sortiert vorliegenden Liste von Daten werden zum Beispiel die oberen und unteren $5\%$ der Daten abgeschnitten, damit fallen auch eventuelle Ausreißer raus. Die Datensatzgröße bleibt jedoch nicht erhalten.

\paragraph{Das winsorisierte Mittel}
Ähnlich wie beim getrimmten Mittel wird der Datensatz beim winsorisierten Mittel von oben und unten herein bearbeitet. Anstatt Daten zu löschen werden beispielsweise die oberen $5\%$ durch den nächstkleineren Wert ersetzt. Hierbei bleibt also die Datensatzgröße gleich.

\subsection{Median}
Der Median stellt ein robusteres Lagemaß als das arithmetische Mittel dar, er ist resistenter gegen Ausreißer im Datensatz.
Für $x_1\leq x_2\leq\ldots\leq x_n$, also einen sortiert vorliegenden Datensatz ist der Median
\begin{equation*}
	x_{\operatorname{med}}=
	\begin{cases}
		x_{\frac{n+1}2}& \text{$n$ ungerade}\\
		\frac 12 (x_{\frac n2}+x_{\frac n2 +1})& \text{$n$ gerade}\\
	\end{cases}
\end{equation*}
Benötigt eine Zahlenordnung, also eine Ordinalskala.

Der Modus verhält sich unter linearer Transformation $y=ax+b$ genauso wie das arithmetische Mittel $y_{\operatorname{med}}=ax_{\operatorname{med}}+b$.

Mindestens $50\%$ der Daten sind kleiner oder gleich $x_{\operatorname{med}}$, genauso sind mindestens $50\%$ der Daten größer oder gleich dem Median $x_{\operatorname{med}}$.

\subsection{Modus}
Der Modus ist die Ausprägung größter Häufigkeit $x_{\operatorname{mod}}=a_i$ mit $h(a_i)=\max \set{h(a)}{a\in A}$ wobei $A$ die Menge aller vorkommenden Ausprägungen der Urliste ist.
Der Modus ist dann eindeutig, wenn die Häufigkeitsverteilung ein eindeutiges Maximum besitzt.

Der Modus empfiehlt sich schon für nominalskalierte Daten.

Der Modus verhält sich unter linearer Transformation $y=ax+b$ genauso wie das arithmetische Mittel $y_{\operatorname{mod}}=ax_{\operatorname{mod}}+b$.



\subsection{Geometrisches Mittel}
Für eine Urliste $U=\simpleset{u_1,\ldots,u_n}$ ist das geometrische Mittel definiert als
\begin{equation*}
	x_{\operatorname{geom}}=\sqrt[n]{\prod_{i=1}^n u_i}.
\end{equation*}
Das geometrische Mittel wird z.B. bei der Berechnung des effektiven Jahreszinses verwendet, es stellt jedoch kein Lagemaß im engeren Sinne dar.

\subsection{Harmonisches Mittel}
Für eine Urliste $U=\simpleset{u_1,\ldots,u_n}$ ist das harmonische Mittel definiert als
\begin{equation*}
	x_{\operatorname{harm}}=\frac{1}{\frac 1n\sum_{i=1}^n \frac 1{x_i}}.
\end{equation*}
Genauso wie das geometrische Mittel zählt das harmonische nicht zu den Lagemaßen im engeren Sinne.



\section{Lageregeln}
Für symmetrische Verteilungen gilt $\overline x\approx x_{\operatorname{med}}\approx x_{\operatorname{mod}}$.

Für linkssteile Verteilungen $\overline x> x_{\operatorname{med}}> x_{\operatorname{mod}}$.

Und ebenso für rechtssteile Verteilungen $\overline x< x_{\operatorname{med}}< x_{\operatorname{mod}}$.
